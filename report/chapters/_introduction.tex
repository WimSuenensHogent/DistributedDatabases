
\chapter{Introduction}
\newcommand{\linkCourse}{https://bamaflexweb.hogent.be/BMFUIDetailxOLOD.aspx?a=140461&b=5&c=2}
\newcommand{\linkAutoSklearn}{https://automl.github.io/auto-sklearn/master/}
\newcommand{\linkKaggle}{https://www.kaggle.com/}
\newcommand{\linkJupyter}{https://jupyter.org/}
\newcommand{\linkLatex}{https://www.latex-project.org/}
\newcommand{\linkGithubRepo}{https://github.com/WimSuenensHogent/DistributedDatabases}

This paper is the outcome of our coursework assignment for the course \href{\linkCourse}{'Distributed Databases'\footnote{\url{\linkCourse}}} taught by Professor Stijn Lievens.
\\

The purpose of the assignment is to get familiar with the \href{\linkAutoSklearn}{'auto-sklearn'\footnote{\url{\linkAutoSklearn}}} toolkit and build and analyze different regression and classifcation models. The datasets used are sourced either from \href{\linkKaggle}{Kaggle\footnote{\url{\linkKaggle}}} or are provided by the instructor during one of the regular courses.
\\

The tools we used are:
\begin{itemize}
  \item \href{\linkJupyter}{Jupyter Notebook\footnote{\url{\linkJupyter}}}
  \item \href{\linkLatex}{\LaTeX{}\footnote{\url{\linkLatex}}}
\end{itemize}

Our notebooks can be found on this \href{\linkGithubRepo}{GitHub repository\footnote{\url{\linkGithubRepo}}}. Cloning the repository and running $`pipenv\ install'$, will prepare your local environment. By running $`jupyter\ notebook'$ within the newly created virtual environment, a local kernel will start on localhost where you can interact with the notebooks.
