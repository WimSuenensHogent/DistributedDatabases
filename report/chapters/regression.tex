\chapter{Regression}
In this section we discuss the datasets we used to research regression.
We will explain in greater detail how our model performed, compare it to other models on and give some info over the dataset.

\section{Used Car Prices}
For this dataset we used the competition version of the big data set provided on Kaggle. This set is based on a big dataset of 100 000 listings on used cars in the UK. We then found and used a version that was slimmed down for competition. This dataset looked ideal for us to learn more about auto-sklearn.
\\
\url{https://www.kaggle.com/adityadesai13/used-car-dataset-ford-and-mercedes?select=vw.csv}
\url{https://www.kaggle.com/kukuroo3/used-car-price-dataset-competition-format}

\subsection{Dataset}

The dataset is based on a list of 100 000 listings of used cars in the UK.
This dataset was preprocessed and separated into competition format.
The label for the test data is provided in the form of a function.
After cleaning the data we where left with the following columns:
\\
carID, brand ,model year, transmission, mileage, fuelType, tax, mpg, engineSize, price.
Their are 2672 entries in total.
\subsection{Model Performance}

First we visualised the dataset with 2 different pairplots.
This gave us an idea of what to expected from this dataset.
The pairplot that is based on the fueltype shows clearly that petrol cars are the most common.
\begin{figure}
    \includegraphics[width=\linewidth]{images/pairplot_fueltype.png}
    \caption{A pairplot based on fueltype}
    \label{fig:pairplot fueltype}
\end{figure}
The pairplot that is based on the brand shows clearly that the german brands are most common.
\begin{figure}
    \includegraphics[width=\linewidth]{images/pairplot_brand.png}
    \caption{A pairplot based on brand}
    \label{fig:pairplot brand}
\end{figure}
\\
After fitting and training the data this are the results we got for the different data sets.

\begin{table}[h!]
    \begin{center}
        \caption{Table with training results.}
        \label{tab:training results}
        \begin{tabular}{l|S|r|l} % <-- Changed to S here.
            \textbf{function} & \textbf{Training data} & \textbf{Test data} & \textbf{Validation data}\\
            \hline
            mean squared error & 59050.4794 & 293230.5317 & 585230.4846\\
            mean absolute error & 26.3980 & 55.7094 & 59.8129\\
            r2 & 0.9998 & 0.9989 & 0.9979\\
        \end{tabular}
    \end{center}
\end{table}

\subsection{Model Comparision}

We chose 3 models from kaggle to compare our results with. After comparing our scores we found out that we have a better performing model. Our model included te model of the car, this was something the other models left out. We can conclude that our models performs better than the ones we compared scores with.
\\
Url1: \url{https://www.kaggle.com/selinsong/u-car-rf-r2-score-0-94/data}

Url2: \url{https://www.kaggle.com/yuyuyuyuy/rf-test-r2-score-0-956}

Url3: \url{https://www.kaggle.com/johyunkang/py-rf-test-r2-0-939}

\begin{table}[h!]
    \begin{center}
        \caption{Table with r2 results.}
        \label{tab:training results}
        \begin{tabular}{l|S} % <-- Changed to S here.
            \textbf{data} & \textbf{R2 score} \\
            \hline
            Training data & 0.9997784238195022\\
            Test data & 0.9989334899119373 \\
            Validation data & 0.9978756073515518\\
            Url1 & 0.94\\
            Url2 & 0.956\\
            Url3 & 0.939\\
        \end{tabular}
    \end{center}
\end{table}

\section{Television Brand E-commerce}
This dataset can be used to explore the current market scenario for Televisions. There are various types of screens with different operating systems offered by several manufacturers at competitive prices.

\subsection{Dataset}
This dataset includes specifications of different televisions offered by various brands with prices and ratings.
It contains 912 samples with 7 attributes. Here are the columns in this dataset-
Brand Resolution Size Selling Price Original Price Operating System Rating

Brand: This indicates the manufacturer of the product i.e. Television
Resolution: This has multiple categories and indicates the type of display i.e. LED, HD LED, etc.
Size: This indicates the screen size in inches
Selling Price: This column has the Selling Price or the Discounted Price of the product
Original Price: This includes the Original Price of the product from the manufacturer.
Operating system: This categorical variable shows the type of OS like Android, Linux, etc.
Rating: Average customer ratings on a scale of 5.

\subsection{Model Performance}
First we visualised the dataset with a pairplot.
This gave us an idea of what to expected from this dataset.
The pairplot that is based on the resolution shows clearly HD LED and ULTRA HD LED are the most common resolution.
\begin{figure}
    \includegraphics[width=\linewidth]{images/pairplot_resolution.png}
    \caption{A pairplot based on tv resolution}
    \label{fig:pairplot resolution}
\end{figure}

After fitting and training the data this are the results we got for the different data sets.

\begin{table}[htp]
    \begin{center}
        \caption{Table with training results.}
        \label{tab:training results}
        \begin{tabular}{l|S|r} % <-- Changed to S here.
            \textbf{function} & \textbf{Training data} & \textbf{Test data} \\
            \hline
            mean squared error & 81016219.59417547 & 952681194.3707451 \\
            mean absolute error & 5251.913730638445 & 14205.972881317139 \\
            r2 & 0.9796709758590545 & 0.8247644898857763\\
        \end{tabular}
    \end{center}
\end{table}

\subsection{Model Comparision}
When we looked on the kaggle page of this dataset we found out that there are not yet any other auto-sklearn projects submitted. Thats why we can not compare our results with someone else. We do notice te big difference between test and training data results.
Url1: \url{https://www.kaggle.com/devsubhash/television-brands-ecommerce-dataset/code}